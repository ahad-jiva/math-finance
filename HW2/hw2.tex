\documentclass{article}
\usepackage{graphicx} % Required for inserting images
\usepackage{amsmath}
\usepackage{mathalfa}
\usepackage{blindtext}
\usepackage[letterpaper, portrait, margin=0.75in]{geometry}
\usepackage{amssymb}
\usepackage{epsf, subfigure, verbatim, epsfig}
\usepackage{fancyhdr}
\usepackage{calc}
\usepackage{ifthen}
\usepackage{layout}
\usepackage{fancybox}
\usepackage{eurosym}
\usepackage{tabularx}
\usepackage{xspace}
\usepackage{dsfont,mathrsfs}
\usepackage{amssymb}
\usepackage{theorem}
\usepackage{multicol}
\usepackage{float}
\usepackage{tikz}
\usepackage{pgfplots}
\pgfplotsset{compat=1.18}

\title{Homework 2 \\ \large MATH 476}
\author{Ahad Jiva}
\date{April 15, 2025}

\begin{document}

\maketitle
\section*{Exercise 17} 

An investor sells a European call option with strike price K and maturity T, and buys a put with
the same strike price and maturity. Describe the investor’s position–describe all of the possible situations at
maturity, explain which (if any) of the options the investor should exercise at maturity, and find the investor’s
payoff in each case. \\

\textbf{Solution:}
The investor has a short position on an ECO and a long position on an EPO. There are two cases to consider:
\begin{enumerate}
    \item The long position will exercise if $S_T > K$. Thus, the investor will have to sell to the long position for $K$. At time $t=T$, the payoff is \\
    
    $\begin{cases}
        0 & S_T \leq K \\
        K - S_T & S_T > K
    \end{cases}$

    or $-\text{max}\{S_T-K, 0\}$.

    \item The investor will exercise their long position on the EPO if $S_T < K$. At time $t=T$ the payoff is \\
    
    $\begin{cases}
        K-S_T & S_T < K \\
        0 & S_T \geq K
    \end{cases}$

    or $-\text{max}\{K-S_T, 0\}$.
\end{enumerate}
Then the overall payoff at expiry will be 
\begin{center}
    $-\text{max}\{S_T-K, 0\} + \text{max}\{K-S_T, 0\} = 
    \begin{cases}
        0 + K - S_T & S_T < K \\
        K - S_T + 0 & S_T \geq K
    \end{cases}
    = K - S_T$.
\end{center}
The payoff diagram looks like the following: \\

\begin{center}
    \begin{tikzpicture}
        \draw[black, thick] [->] (-5,0) -- (-5, 5);
        \draw[black, thick] [->] (-5,2) -- (0,2);
        \filldraw[black] (-5,4) circle (2pt) node[anchor=east]{K};
        \filldraw[black] (-3,2) circle (2pt) node[anchor=north]{K};
        \draw[gray, thick] [->] (-5,4) -- (-2,1);
        \filldraw[black] (0,2) circle (0pt) node[anchor=west]{$S_T$};
        \filldraw[black] (-5,5) circle (0pt) node[anchor=south]{Payoff};
    \end{tikzpicture}
\end{center}
\break

\section*{Exercise 18}
A trader buys a call option with a strike price of \$45 and a put option with a strike price of \$40.
Both options have the same maturity. The call costs \$3 and the put costs \$4. Draw a diagram showing the
variation of the trader’s profit with the asset price. Note: this type of trading strategy is known as a strangle. \\

\textbf{Solution:} Let's start by calculating the profit for each option: \\
\begin{enumerate}
    \item Long Position on an ECO: \\
    $K = \$45$, $c = \$3$, expiry $t=T$. The profit is represented as
    \begin{center}
        $\begin{cases}
            S_T - 45 - 3 & S_T \geq 45 \\
            -3 & S_T < 45
        \end{cases}$
    \end{center}
    
    \item Long Position on an EPO: \\
    $K = \$40$, $c = \$4$, expiry $t=T$. The profit is represented as
    \begin{center}
        $\begin{cases}
            40-4-S_T & S_T < 40 \\
            -4 & S_T \geq 40
        \end{cases}$
    \end{center}
\end{enumerate}
Then the net profit is 
\begin{center}
    $\begin{cases}
        36-S_T-3 & S_T < 40 \\
        -3-4 & 40 \leq S_T \leq 45 \\
        S_T - 48 - 4 & S_T > 45
    \end{cases}$
\end{center}
The payoff diagram is as follows:
\begin{center}
    \begin{tikzpicture}
        \begin{axis}[xmin=0, xmax=55, ymin=-10, ymax=35, axis lines=middle];
            \addplot[domain=0:40]{33-x};
            \addplot[domain=40:45]{-7};
            \addplot[domain=45:55] [->] {x-52};
        \end{axis};
        \filldraw[black] (7,1) circle (0pt) node[anchor=west]{$S_T$};
        \filldraw[black] (0,6) circle (0pt) node[anchor=east]{Profit};
    \end{tikzpicture}
\end{center}

\section*{Exercise 26}
On May 21, 2020, an investor owns 100 Apple shares. The
investor is comparing two alternatives to limit risk. The first involves buying one December put option contract
with a strike price of \$290. The second involves instructing a broker to sell the 100 shares as soon as Apple’s
price reaches \$290. Discuss the advantages and disadvantages of the two strategies. \\

\textbf{Solution:} Let's look at the two possible choices separately:
\begin{enumerate}
    \item Buying a put option: $K = \$290, c = \$21.30$. Then the profit is
    \begin{center}
        $\begin{cases}
        -(21.30)(1000) & S_T > K \\
        K - S_T - 2130 & S_T \leq K
        \end{cases}$
    \end{center}
    \item Use a stop-loss to sell AAPL at \$290. This option is less risky, but it could lose out on profits if the market is volatile.
\end{enumerate}
Using a stop-loss is a less risky and simpler strategy that will ensure an investor will receive an exact amount for their shares. On the other hand,
using a put option could allow the investor to make a profit in the future if the stock price is predicted to decline. Given that the investor holds shares
at this instant, using a put option allows them to spend a portion of the value of the shares as "insurance", bounding the maximum possible loss. Also, if the market
is volatile, the stop-loss could be triggered before a breakout rally, causing the investor to lose out on potential profits.

\end{document}
