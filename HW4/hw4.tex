\documentclass{article}
\usepackage{graphicx} % Required for inserting images
\usepackage{amsmath}
\usepackage{mathalfa}
\usepackage{blindtext}
\usepackage[letterpaper, portrait, margin=0.75in]{geometry}
\usepackage{amssymb}
\usepackage{epsf, subfigure, verbatim, epsfig}
\usepackage{fancyhdr}
\usepackage{calc}
\usepackage{ifthen}
\usepackage{layout}
\usepackage{fancybox}
\usepackage{eurosym}
\usepackage{tabularx}
\usepackage{xspace}
\usepackage{dsfont,mathrsfs}
\usepackage{amssymb}
\usepackage{theorem}
\usepackage{multicol}
\usepackage{float}

\title{Homework 4 \\ \large MATH 476}
\author{Ahad Jiva}
\date{May 2, 2025}

\begin{document}

\maketitle
\section*{Exercise 45}
\begin{flushleft}
    $S_0 = 19$, $c=1$, $K=20$, $r=0.04$, expiry in 3 months means $T = 1/4$.
    We can use put-call parity to calculate the price of a put option with same strike price and expiry.
    \begin{center}
        $c + Ke^{-rT} = p + S_0$ \\
        $1 + 20e^{-0.04 \cdot 0.25} = p + 19$ \\
        $p = 1 + 20e^{-0.04 * 0.25} - 19$ \\
        $p = 1.80$
    \end{center}
\end{flushleft}

\section*{Exercise 46}
\begin{flushleft}
    $S_0 = 130$, expiry in one year means $T=1$, $c = 20$, $p=5$, $K = 120$.
    We can use put-call parity to calculate the risk-free interest rate.
    \begin{center}
        $c + Ke^{-rT} = p + S_0$ \\
        $20 + 120e^{-r \cdot 1} = 5 + 130$ \\
        $e^{-r} = \frac{5+130-20}{120}$ \\
        $e^r = \frac{120}{115}$ \\
        $r = \ln(\frac{120}{115}) = 0.0426$ \\
    \end{center}
    Thus the risk-free interest rate is 4.26\%.
\end{flushleft}

\section*{Exercise 47}
\begin{flushleft}
    $S_0 = 31$, $c = 3$, $p = 2.25$, $K = 30$, $T = 0.25$, $r=0.1$.
    Note that put-call parity does not hold here:
    \begin{center}
        $c + Ke^{-rT} = p + S_0$ \\
        $3 + 30e^{-0.1 \cdot 0.25} = 2.25 + 31$ \\
        $32.26 \neq 33.25$
    \end{center}
    Thus we should be able to construct an arbitrage opportunity. Consider a portfolio where
    we buy the call option and short-sell the put option and the stock. At $t=0$, the cash flow is
    \begin{center}
        $-c + p + S_0 = \$30.25$
    \end{center}
    Thus we have positive cash flow at $t=0$. We can then invest this at the risk-free interest rate. At expiry, this will be worth \$31.02. At expiry, we have two cases.
    \begin{enumerate}
        \item $S_T \leq 30$. Then we let the ECO expire, the EPO will be exercised, and we return the stock we shorted. The payoff will be $0 - 30 = -30$,
                and since we started with \$31.02, we have \$1.02 profit.
        \item $S_T > 30$. Then we exercise the ECO, the EPO expires worthless, and we return the stock we shorted. The payoff is $-K + S_T - S_T = -30$.
                Since we started with \$31.02 we have \$1.02 profit.
    \end{enumerate}
    Thus in all cases, we make a profit with positive initial cash flow, and this was an arbitrage opportunity. $\square$
\end{flushleft}

\section*{Exercise 54}
We have three call options with prices $c(K_1), c(K_2), c(K_3)$ with $K_1 < K_2 < K_3$.
We also have three put options with prices $p(K_1), p(K_2), p(K_3)$. All of these options have the same expiry.
\begin{enumerate}
    \item Show $c(K_1) \geq c(K_2)$. Suppose not for the sake of contradiction, ie $c(K_1) < c(K_2)$.
        Consider the portfolio where we are short in $c(K_2)$ and long in $c(K_1)$. The initial cash flow is $c(K_2) - c(K_1) > 0$, and let $CF_0$ represent this amount.
        \begin{enumerate}
            \item Case 1: $S_T < K_1 < K_2$. Then none of the call options will be exercised. The payoff is zero and the profit is $0 + CF_0 \cdot e^{rT}$, which is positive.
            \item Case 2: $K_1 < S_T < K_2$. Then the $c(K_1)$ option will be exercised, which is our long position. So the payoff is $S_T - K_1$ and the profit is $S_T - K_1 + CF_0 \cdot e^{rT}$, which is positive.
            \item Case 3: $K_1 < K_2 < S_T$. Then both options will be exercised. The payoff will be $S_T - K_1 - S_T + K_2 = K_2 - K_1$, which is positive, and thus the profit is $K_2 - K_1 + CF_0 \cdot e^{rT}$, which is positive.
        \end{enumerate}
        Thus in all cases, we have a risk free profit with no initial investment, ie an arbitrage opportunity. Contradition. Thus $c(K_1) \geq c(K_2)$. $\square$
    \item Show $p(K_2) \geq p(K_1)$. Suppose not for the sake of contradiction, ie $p(K_2) < p(K_1)$ or $p(K_1) - p(K_2) > 0$.
        Consider the portfolio where we are short in $p(K_1)$ and long in $p(K_2)$. Then the initial cash flow is $p(K_1) - p(K_2) > 0$, and let $CF_0$ represent this amount.
        \begin{enumerate}
            \item Case 1: $S_T < K_1 < K_2$. Then both put options will be exercised. The payoff is $S_T - K_1 + K_2 - S_T = K_2 - K_1$. So the profit is $K_2 - K_1 + CF_0 \cdot e^{rT}$, which is positive.
            \item Case 2: $K_1 < S_T < K_2$. Then the $p(K_2)$ option will be exercised. The payoff is $K_2 - S_T$ and the profit is $K_2 - S_T + CF_0 \cdot e^{rT}$, which is positive.
            \item Case 3: $K_1 < K_2 < S_T$. Then none of the put options will be exercised and the payoff is zero. Then the profit is $CF_0 \cdot e^{rT}$, which is positive.
        \end{enumerate}
        Thus in all cases, we have a risk free profit with no initial investment, ie an arbitrage opportunity. Contradiction. Thus $p(K_2) \geq p(K_1)$. $\square$
    \item Show $c(K_1) - c(K_2) \leq K_2 - K_1$. Suppose not for the sake of contradiction, ie $c(K_1) - c(K_2) > K_2 - K_1$.
        Consider the portfolio where we are short in $c(K_1)$ and long in $c(K_2)$, and also we hold cash equivalent to $(K_1 - K_2) \cdot e^{-rT}$. Then the initial cash flow is $c(K_1) - c(K_2) + (K_1 - K_2)\cdot e^{-rT}$, which we denote as $CF_0$, which is positive.
        \begin{enumerate}
            \item Case 1: $S_T < K_1 < K_2$. Then neither call option is exercised. The payoff is zero, and the profit is $CF_0 \cdot e^{rT}$.
            \item Case 2: $K_1 < S_T < K_2$. Then $c(K_1)$ is exercised. The payoff is $K_1 - S_T$, and thus the profit is $K_1 - S_T + CF_0 \cdot e^{rT}$, which is positive.
            \item Case 3: $K_1 < K_2 < S_T$. Then both call options are exercised. The payoff is $K_2 - K_1$, which when added to $CF_0\cdot e^{rT}$ to get profit is positive.
        \end{enumerate}
        Thus in all cases, we have a risk free profit with no initial investment, ie an arbitrage opportunity. Contradition. Then $c(K_1) - c(K_2) \leq K_2 - K_1$. $\square$
    \item Show $p(K_2) - p(K_1) \leq K_2 - K_1$. Suppose not for the sake of contradiction, ie $p(K_2) - p(K_1) > K_2 - K_1$.
        Consider the portfolio where we are short in $p(K_1)$ and long in $p(K_2)$, and also we hold cash equivalent to $(K_1 - K_2) \cdot e^{-rT}$. Then the initial cash flow is $p(K_1) - p(K_2) + (K_1 - K_2) \cdot e^{-rT}$, denoted as $CF_0$, which is positive.
        \begin{enumerate}
            \item Case 1: $S_T < K_1 < K_2$. Then both put options will be exercised. The payoff will be $K_2 - K_1$, which when added to $CF_0 \cdot e^{rT}$ to get profit, gives us a positive value.
            \item Case 2: $K_1 < S_T < K_2$. Then $p(K_2)$ is exercised. The payoff will be $K_2 - S_T$, which is positive. When added to $CF_0 \cdot e^{rT}$, we get a positive profit.
            \item Case 3: $K_1 < K_2 < S_T$. Then neither put option will be exercised. The payoff is zero and the total profit is $CF_0 \cdot e^{rT}$ which is positive.
        \end{enumerate}
        Thus in all cases we get a risk free profit will no initial investment, ie an arbitrage opportunity. Contradiction. Then $p(K_2) - p(K_1) \leq K_2 - K_1$. $\square$
\end{enumerate}

\end{document}
